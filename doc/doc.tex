\documentclass{article}
\usepackage{graphicx}
\usepackage{hyperref}
\usepackage[czech]{babel}
\usepackage{csquotes}

\usepackage{biblatex}
\addbibresource{doc.bib}

% Title, Author, Date
\title{IMS T6: Komunikace a služby}
\author{
    Martin Slezák\\
    \texttt{xsleza26}
    \and
    Jakub Antonín Štigler\\
    \texttt{xstigl00}
}
\date{\today}

\begin{document}

\maketitle

\newpage
\tableofcontents

\newpage
\section{Úvod}

Tento projekt představuje simulaci historického modelu telekomunikační
infrastruktury\cite{wikipedia}, konkrétně v období, kdy byly telefonní hovory
zajišťovány manuálními ústřednami. Tyto ústředny zajišťovaly propojení hovorů
mezi účastníky. Každá ústředna obsahovala několik stanic se zásuvnými deskami,
které byly vybaveny konektory pro připojení jednotlivých telefonních spojení.
U těchto stanic pracovali operátoři, jejichž úkolem bylo manuálně připojovat
dráty do správných konektorů, aby bylo možné uskutečnit hovor.

\subsection{Průběh hovoru}

Proces spojení hovoru probíhal následujícím způsobem:

\begin{enumerate}
    \item \textbf{Inicializace hovoru:} Jakmile volající zvedl sluchátko, na
        stanici operátora se rozsvítila dioda signalizující příchozí hovor a
        upozornila ho na potřebnou akci.
    \item \textbf{Dotaz na číslo:} Operátor připojil odpovědní drát do
        příslušného konektoru a zahájil komunikaci s volajícím a zeptal se na
        požadované telefonní číslo.
    \item \textbf{Rozřazení hovoru:} Na základě čísla operátor určil, zda jde
        o:
        \begin{itemize}
            \item \textbf{Lokální hovor:} Operátor připojil vyzváněcí drát k
                cílovému konektoru, což aktivovalo zvonění na straně
                volajícího.
            \item \textbf{Lokální hovor na jinou desku:} Pokud bylo volané
                číslo na jiné desce, operátor přesměroval hovor v rámci
                ústředny.
            \item \textbf{Dálkový hovor:} V případě dálkových hovorů
                operátor využil kmenovou síť, aby propojil jiného operátora v
                jiné ústředně.
        \end{itemize}
\end{enumerate}

\subsection{Výsledek modelu}

Výsledek modelu nám ukáže, jak velký čas pracovníci reálně pracují a také
jestli ústředna stíhá všechny hovory, jak má. Na základě tohoto modelu můžeme
poté vidět, jak efektivní je práce a zda daná ústředna má dost úředníku a
jednotlivých stanic. Také se dozvíme základní statistiky o hovorech, jako je
průměrný čas čekání na spojení, průměrný čas zvonění nebo průměrný čas hovoru.

\section{Fakta a hypotézy}

Při tvoření modelu jsme se snažili odhadnout ceny jednotlivých přechodů na
základě našeho internetového průzkumu. Některé přechody jsou časové, některé
zase náhodné s nepravidelným rozložením. V této sekci popíšeme chtěl popsat ty
nejzajímavější.

\subsection{Fakta}

V této sekci jsou popsány informace, které máme podloženy na základě
internetového průzkumu.

\subsubsection*{Doba spojení dálkového hovoru}

Doba spojení u dálkových hovorů se mohla velmi lišit v závislosti na dálce
spojení a počtu ústředen na trase hovoru. V roce 1918 byl průměrný čas spojení
dálkového hovoru 15 minut\cite{wikipedia}. My jsme tedy pro náš model zvolili
exponencionální rozložení se středem 20 minut.

\subsection{Hypotézy}

Mnoho údajů z dřívejší doby ohledně telefoniky není bohužel dohledatelných,
tudíž jsme některé informace museli odhadnout.

\subsubsection*{Dotaz na číslo}

Například dotaz na číslo jsme odhadli, že bude trvat zhruba 20-60 sekund.
V případě přesměrování se poté jednotlivý operátoři ptají na číslo a jelikož
již mají praxi, zabere jim to pouze 10-20 sekund. Tento odhad vznikl na základě
simulace situace, kdy jsme se zkusili zeptat na číslo a zohlednili jsme, že
někteří volající můžou mít například problém se čtením a bude jim trvat číslo
přečíst.

\subsubsection*{Doba hovoru}

Dříve hovory nebyly tak časté a když už někdo někomu volal, pravděpodobně s ním
nějakou dobu mluvil. Na základě tohoto úsudku jsme se rozhodli modelovat dobu
hovoru jako exponencionální rozložení se středem 5 minut. Tento odhad vznikl
na základě porovnání vůči dnešní době, kdy většina hovorů trvá pouze několik
sekund, zato však probíhají klidně i několikrát denně. Když by se tedy tyto
hovory nahradili například jedním hovorem za týden, trval by několik minut.

\subsubsection*{Dělení hovoru}

Příchozí hovor náhodně dělíme do čtyř podkategorií s nepravidelným rozložením.
Tyto kategorie jsou:
\begin{enumerate}
    \item Neznámé číslo (5\%)
    \item Přímé spojení (70\%)
    \item Dálkové spojení (10\%)
    \item Přesměrování na jinou desku (15\%)
\end{enumerate}

Tento odhad je založen na předpokladu, že volající bude bydlet v okolí svých
příbuzných a tedy většina hovorů bude přímé spojení.

\section{Spuštění projektu}

\subsection*{Nainstalování závislostí}

Před samotným spuštěním je třeba mít nainstalované tyto závislosti:
\begin{itemize}
    \item make
    \item cmake
    \item C++ compiler pro C++ 20 a novější
    \item SIMLIB
\end{itemize}

\subsection*{Spuštění}

Projekt lze spustit pomocí připraveného \texttt{Makefile}:
\begin{verbatim}
    make run
\end{verbatim}

\noindent Je také možné projekt pouze zkompilovat:
\begin{verbatim}
    make
\end{verbatim}

\subsection*{Konfigurace}

Náš program podporuje i nakonfigurovat proměnné dle potřeby. Jednotlivé
parametry lze nastavit za pomocí argumentů při spuštění aplikace. Více se lze
dozvědět v nápovědě programu.
\begin{verbatim}
    ./ims-t6 -h
\end{verbatim}

\newpage
\nocite{*}
\printbibliography

\end{document}

